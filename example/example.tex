\documentclass[12pt]{article}
\usepackage[utf8]{inputenc}
\usepackage{parskip}
\usepackage{soul}
\usepackage{amsmath}
\usepackage{amssymb}
\usepackage{braket}
\usepackage{fancyhdr}

\pagestyle{fancy}
\setlength{\headheight}{15pt}
\setlength{\headsep}{1.5cm}
\fancyhead{}
\fancyhead[L]{{\small All The Good Things}}
\fancyhead[R]{{\small Valentine's Day 2023}}

\begin{document}

\begin{center}
{\Huge Example}
\end{center}
\vspace{1cm}

\begin{center}
Testing random stuff: \textbf{bold}, \textit{italics}, \st{strike through}, `singly quoted strings', ``doubly quoted strings'', \%, and \#. --~\textit{Aditya}

\vspace{1cm}

In many ways, the work of a critic is easy. We risk very little, yet enjoy a position over those who offer up their work and their selves to our judgment. We thrive on negative criticism, which is fun to write and to read. But the bitter truth we critics must face, is that in the grand scheme of things, the average piece of junk is probably more meaningful than our criticism designating it so. But there are times when a critic truly risks something, and that is in the discovery and defense of the \textbf{new}. The world is often unkind to new talent, new creations. The new needs friends. Last night, I experienced something new: an extraordinary meal from a singularly unexpected source. To say that both the meal and its maker have challenged my preconceptions about fine cooking is a gross understatement. They have rocked me to my core. In the past, I have made no secret of my disdain for Chef Gusteau's famous motto, ``Anyone can cook.'' But I realize, only now do I truly understand what he meant. Not everyone can become a great artist; but a great artist \textit{can} come from \textbf{anywhere}. It is difficult to imagine more humble origins than those of the genius now cooking at Gusteau's, who is, in this critic's opinion, nothing less than the finest chef in France. I will be returning to Gusteau's soon, hungry for more. --~\textit{Aditya}

\vspace{1cm}

I love writing. When I was younger, I remember reading this quote from Gary Provost, which has really shaped my writing style: ``This sentence has five words. Here are five more words. Five-word sentences are fine. But several together become monotonous. Listen to what is happening. The writing is getting boring. The sound of it drones. It’s like a stuck record. The ear demands some variety. Now listen. I vary the sentence length, and I create music. Music. The writing sings. It has a pleasant rhythm, a lilt, a harmony. I use short sentences. And I use sentences of medium length. And sometimes when I am certain the reader is rested, I will engage him with a sentence of considerable length, a sentence that burns with energy and builds with all the impetus of a crescendo, the roll of the drums, the crash of the cymbals–sounds that say listen to this, it is important. So write with a combination of short, medium, and long sentences. Create a sound that pleases the reader’s ear. Don’t just write words. Write music.'' --~\textit{Aditya}

\vspace{1cm}

In my opinion, one of the most beautiful pieces of writing in the Harry Potter series is the following quote: ``But they were not living, thought Harry: They were gone. The empty words could not disguise the fact that his parents' moldering remains lay beneath snow and stone, indifferent, unknowing. And tears came before he could stop them, boiling hot then instantly freezing on his face, and what was the point in wiping them off or pretending? He let them fall, his lips pressed hard together, looking down at the thick snow hiding from his eyes the place where the last of Lily and James lay, bones now, surely, or dust, not knowing or caring that their living son stood so near, his heart still beating, alive because of their sacrifice and close to wishing, at this moment, that he was sleeping under the snow with them.'' Harry and Herminoe's visit to Godric's Hollow is full of deeply emotional passages like this one. --~\textit{Aditya}

\vspace{1cm}

I still don't really understand what a tensor is. Stuff like $$\int_{\partial A} \frac{s_i s_j s_k s_m n_k}{s^6} dA$$ or whatever makes no sense to me. What is $s_i$???? --~\textit{Aditya}

\vspace{1cm}

\end{center}

\end{document}